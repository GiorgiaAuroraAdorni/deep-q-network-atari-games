\documentclass[a4paper,12pt]{article} % This defines the style of your paper

\usepackage[top = 2.5cm, bottom = 2.5cm, left = 2.5cm, right = 2.5cm]{geometry} 
\usepackage[utf8]{inputenc} %utf8 % lettere accentate da tastiera
\usepackage[english]{babel} % lingua del documento
\usepackage[T1]{fontenc} % codifica dei font

\usepackage{multirow} % Multirow is for tables with multiple rows within one 
%cell.
\usepackage{booktabs} % For even nicer tables.

\usepackage{graphicx} 

\usepackage{setspace}
\setlength{\parindent}{0in}

\usepackage{float}

\usepackage{fancyhdr}

\usepackage{caption}
\usepackage{amssymb}
\usepackage{amsmath}
\usepackage{mathtools}
\usepackage{color}

\usepackage[hidelinks]{hyperref}
\usepackage{csquotes}
\usepackage{subfigure}

\usepackage{ifxetex,ifluatex}
\usepackage{etoolbox}
\usepackage[svgnames]{xcolor}

\usepackage{tikz}

\usepackage{framed}

 \newcommand*\quotefont{\fontfamily{LinuxLibertineT-LF}} % selects Libertine as 
 %the quote font


\newcommand*\quotesize{40} % if quote size changes, need a way to make shifts 
%relative
% Make commands for the quotes
\newcommand*{\openquote}
{\tikz[remember picture,overlay,xshift=-4ex,yshift=-1ex]
	\node (OQ) 
	{\quotefont\fontsize{\quotesize}{\quotesize}\selectfont``};\kern0pt}

\newcommand*{\closequote}[1]
{\tikz[remember picture,overlay,xshift=4ex,yshift=-1ex]
	\node (CQ) {\quotefont\fontsize{\quotesize}{\quotesize}\selectfont''};}

% select a colour for the shading
\colorlet{shadecolor}{WhiteSmoke}

\newcommand*\shadedauthorformat{\emph} % define format for the author argument

% Now a command to allow left, right and centre alignment of the author
\newcommand*\authoralign[1]{%
	\if#1l
	\def\authorfill{}\def\quotefill{\hfill}
	\else
	\if#1r
	\def\authorfill{\hfill}\def\quotefill{}
	\else
	\if#1c
	\gdef\authorfill{\hfill}\def\quotefill{\hfill}
	\else\typeout{Invalid option}
	\fi
	\fi
	\fi}
% wrap everything in its own environment which takes one argument (author) and 
%one optional argument
% specifying the alignment [l, r or c]
%
\newenvironment{shadequote}[2][l]%
{\authoralign{#1}
	\ifblank{#2}
	{\def\shadequoteauthor{}\def\yshift{-2ex}\def\quotefill{\hfill}}
	{\def\shadequoteauthor{\par\authorfill\shadedauthorformat{#2}}\def\yshift{2ex}}
	\begin{snugshade}\begin{quote}\openquote}
		{\shadequoteauthor\quotefill\closequote{\yshift}\end{quote}\end{snugshade}}

\newcommand{\footref}[1]{%
	$^{\ref{#1}}$%
}
\newcommand{\footlabel}[2]{%
	\addtocounter{footnote}{1}%
	\footnotetext[\thefootnote]{%
		\addtocounter{footnote}{-1}%
		\refstepcounter{footnote}\label{#1}%
		#2%
	}%
	$^{\ref{#1}}$%
}


\pagestyle{fancy}

\setlength\parindent{24pt}

\fancyhf{}

\lhead{\footnotesize Deep Learning Lab: Assignment 4}

\rhead{\footnotesize Giorgia Adorni}

\cfoot{\footnotesize \thepage} 

\begin{document}
	\thispagestyle{empty}  
	\noindent{
	\begin{tabular}{p{15cm}} 
		{\large \bf Deep Learning Lab} \\
		Università della Svizzera Italiana \\ Faculty of Informatics \\ \today  \\
		\hline
		\\
	\end{tabular} 
	
	\vspace*{0.3cm} 
	
	\begin{center}
		{\Large \bf Assignment 4: Deep Q-Network}
		\vspace{2mm}
		
		{\bf Giorgia Adorni (giorgia.adorni@usi.ch)}
		
	\end{center}  
}
	\vspace{0.4cm}

	%%%%%%%%%%%%%%%%%%%%%%%%%%%%%%%%%%%%%%%%%%%%%%%%
	%%%%%%%%%%%%%%%%%%%%%%%%%%%%%%%%%%%%%%%%%%%%%%%%
	
	\section{Introduction}
	\label{section:intro}
	
	The goal of this project is to implement and train a Deep Q-Network agent, based on \textit{Mnih et al., 2015}.
	
	All the models were implemented using TensorFlow and trained on an NVIDIA Tesla V100-PCIE-16GB GPU.
	
	\section{Environment}
	\label{section:environment}
	
	
	\texttt{OpenAI Gym} has been used to create the \textit{BreakoutNoFrameskip-v4} environment. Moreover, a seed has been fixed and used for the creation of the environment and for the future random operations. 
	
	\section{Agent and Training}
	\label{section:agent}
	
	The DQN agent has three main components: an online Q-network, a target Q-network and replay buffer.
	The two networks are user to improve the stability of this method, in particular, every $C$ steps, the target network is updated with the online network parameters.
	
	The replay buffer of capacity $10000$ is composed of state, action, reward, next state, and termination flag. It will be used during the training in order to create batches by sampling them from the buffer.
	
	In Table \ref{tab:arc} is summarised the architecture of both the networks. 
	
	\begin{figure}[htb]
		\centering
		
		\begin{tabular}{ccccc}
			\toprule
			\textbf{conv1} & \textbf{conv2} & \textbf{conv3} & \textbf{fc1} &
			\textbf{fc2} \\
			\midrule
			8$\times$8,  32 & 4$\times$4, 64 & 3$\times$3, 64 & 512 & k\\
			s. 4$\times$4 &   s. 2$\times$2 &   s. 1$\times$1 &  & \\
			p. same & p. same & p. same &&\\
			ReLU & ReLU & ReLU & ReLU & ReLU  \\
			\bottomrule
		\end{tabular}
		\captionof{table}{Network architecture}
		\label{tab:arc}
	\end{figure}
	
	
	\section{Tasks}
	\label{section:tasks}
	% FIXME
	1. (10 points) Several Gym wrappers were required to preprocess the data. These wrappers include the FrameStack wrapper, the ScaledFloatFrame wrapper, the MaxAndSkipEnv wrapper, and the ClipRewar- dEnv wrapper. Briefly explain the role of each of these wrappers (in one or two sentences).
	2. (10 points) How do Mnih et al. (2015) explain the need for a target Q-network in addition to an online Q- network?
	3. (10 points) Why is it necessary to act according to an $\epsilon$-greedy policy instead of a greedy policy (with respect to Q)?
	% end FIXME
	
	\subsection{Experiment 1}
	The first experiment includes a training phase the takes $2000000$ steps. 
	Root mean square prop (\texttt{RMSprop}) is used as optimiser.
	
	Figure \ref{fig:step-m1} shows the number of steps elapsed in each episode.
	It is clearly visible how the number of steps per episode increases over time, since the network learns how to play.

	\begin{figure}[htb]
		\centering
		\includegraphics[width=.8\linewidth]{../code/out/m1/img/step-per-episode.pdf}	
		\caption{Steps per episode of the first experiment}
		\label{fig:step-m1}
	\end{figure}
	\bigskip
	%FIXME
	Figure \ref{fig:reward-m1} shows the number of time in which the  of steps elapsed in each episode.
	It is clearly visible how the number of steps per episode increases over time, since the network learns how to play.
	
	\begin{figure}[htb]
		\centering
		\includegraphics[width=.8\linewidth]{../code/out/m1/img/reward.pdf}	
		\caption{Reward obtained at each step in the first experiment}
		\label{fig:reward-m1}
	\end{figure}
	\bigskip

	the reward obtained at each step. 
	 
	% FIXME time
	It is also useful to estimate the remaining training time based on the average time that each step requires.
	
	Figure \ref{fig:movingavg-m1} shows the return per episode, averaged over the last 30 episodes to reduce noise, called moving average.
	
	\begin{figure}[htb]
		\centering
		\includegraphics[width=.8\linewidth]{../code/out/m1/img/moving-average.pdf}	
		\caption{Moving average of the first experiment}
		\label{fig:movingavg-m1}
	\end{figure}
	\bigskip
	Figure \ref{fig:score-m1} shows the scores across 30 independent plays. A score is the sum of the return obtained across a sequence of 5 different episodes that is called play.
	\begin{figure}[htb]
		\centering
		\includegraphics[width=.8\linewidth]{../code/out/m1/img/score.pdf}	
		\caption{Score of the first experiment}
		\label{fig:score-m1}
	\end{figure} 
	\bigskip
	
	At the end, in Figure \ref{fig:loss-m1} is shown the temporal-difference error L($\theta$).
	\begin{figure}[htb]
		\centering
		\includegraphics[width=.8\linewidth]{../code/out/m1/img/loss.pdf}	
		\caption{Loss of the first experiment}
		\label{fig:loss-m1}
	\end{figure} 

\end{document}
