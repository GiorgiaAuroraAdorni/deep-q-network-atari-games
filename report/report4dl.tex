\documentclass[a4paper,12pt]{article} % This defines the style of your paper

\usepackage[top = 2.5cm, bottom = 2.5cm, left = 2.5cm, right = 2.5cm]{geometry} 
\usepackage[utf8]{inputenc} %utf8 % lettere accentate da tastiera
\usepackage[english]{babel} % lingua del documento
\usepackage[T1]{fontenc} % codifica dei font

\usepackage{multirow} % Multirow is for tables with multiple rows within one 
%cell.
\usepackage{booktabs} % For even nicer tables.

\usepackage{graphicx} 

\usepackage{setspace}
\setlength{\parindent}{0in}

\usepackage{float}

\usepackage{fancyhdr}

\usepackage{caption}
\usepackage{amssymb}
\usepackage{amsmath}
\usepackage{mathtools}
\usepackage{color}

\usepackage[hidelinks]{hyperref}
\usepackage{csquotes}
\usepackage{subfigure}

\usepackage{ifxetex,ifluatex}
\usepackage{etoolbox}
\usepackage[svgnames]{xcolor}

\usepackage{tikz}

\usepackage{framed}

 \newcommand*\quotefont{\fontfamily{LinuxLibertineT-LF}} % selects Libertine as 
 %the quote font


\newcommand*\quotesize{40} % if quote size changes, need a way to make shifts 
%relative
% Make commands for the quotes
\newcommand*{\openquote}
{\tikz[remember picture,overlay,xshift=-4ex,yshift=-1ex]
	\node (OQ) 
	{\quotefont\fontsize{\quotesize}{\quotesize}\selectfont``};\kern0pt}

\newcommand*{\closequote}[1]
{\tikz[remember picture,overlay,xshift=4ex,yshift=-1ex]
	\node (CQ) {\quotefont\fontsize{\quotesize}{\quotesize}\selectfont''};}

% select a colour for the shading
\colorlet{shadecolor}{WhiteSmoke}

\newcommand*\shadedauthorformat{\emph} % define format for the author argument

% Now a command to allow left, right and centre alignment of the author
\newcommand*\authoralign[1]{%
	\if#1l
	\def\authorfill{}\def\quotefill{\hfill}
	\else
	\if#1r
	\def\authorfill{\hfill}\def\quotefill{}
	\else
	\if#1c
	\gdef\authorfill{\hfill}\def\quotefill{\hfill}
	\else\typeout{Invalid option}
	\fi
	\fi
	\fi}
% wrap everything in its own environment which takes one argument (author) and 
%one optional argument
% specifying the alignment [l, r or c]
%
\newenvironment{shadequote}[2][l]%
{\authoralign{#1}
	\ifblank{#2}
	{\def\shadequoteauthor{}\def\yshift{-2ex}\def\quotefill{\hfill}}
	{\def\shadequoteauthor{\par\authorfill\shadedauthorformat{#2}}\def\yshift{2ex}}
	\begin{snugshade}\begin{quote}\openquote}
		{\shadequoteauthor\quotefill\closequote{\yshift}\end{quote}\end{snugshade}}

\newcommand{\footref}[1]{%
	$^{\ref{#1}}$%
}
\newcommand{\footlabel}[2]{%
	\addtocounter{footnote}{1}%
	\footnotetext[\thefootnote]{%
		\addtocounter{footnote}{-1}%
		\refstepcounter{footnote}\label{#1}%
		#2%
	}%
	$^{\ref{#1}}$%
}


\pagestyle{fancy}

\setlength\parindent{24pt}

\fancyhf{}

\lhead{\footnotesize Deep Learning Lab: Assignment 4}

\rhead{\footnotesize Giorgia Adorni}

\cfoot{\footnotesize \thepage} 

\begin{document}
	\thispagestyle{empty}  
	\noindent{
	\begin{tabular}{p{15cm}} 
		{\large \bf Deep Learning Lab} \\
		Università della Svizzera Italiana \\ Faculty of Informatics \\ \today  \\
		\hline
		\\
	\end{tabular} 
	
	\vspace*{0.3cm} 
	
	\begin{center}
		{\Large \bf Assignment 4: Deep Q-Network}
		\vspace{2mm}
		
		{\bf Giorgia Adorni (giorgia.adorni@usi.ch)}
		
	\end{center}  
}
	\vspace{0.4cm}

	%%%%%%%%%%%%%%%%%%%%%%%%%%%%%%%%%%%%%%%%%%%%%%%%
	%%%%%%%%%%%%%%%%%%%%%%%%%%%%%%%%%%%%%%%%%%%%%%%%
	
	\section{Introduction}
	\label{section:intro}
	
	The goal of this project is to implement and train a Deep Q-Network agent, based on \textit{Mnih et al., 2015}.
	
	All the models were implemented using TensorFlow and trained on an NVIDIA Tesla V100-PCIE-16GB GPU.
	
	\section{Environment, Agent and Training}
	\label{section:agent}	
	\texttt{OpenAI Gym} has been used to create the \textit{BreakoutNoFrameskip-v4} environment. 
	
	The DQN agent has three main components: an online Q-network, a target Q-network and replay buffer.
	The two networks are used to improve the stability of this method, in particular, every $C$ steps, the target network is updated with the online network parameters.
	
	The replay buffer of capacity $10000$ is composed of state, action, reward, next state, and termination flag. It will be used during the training in order to create batches by sampling them from the buffer.
	
	In Table \ref{tab:arc} is summarised the architecture of both the networks. 
	
	\begin{figure}[htb]
		\centering
		
		\begin{tabular}{ccccc}
			\toprule
			\textbf{conv1} & \textbf{conv2} & \textbf{conv3} & \textbf{fc1} &
			\textbf{fc2} \\
			\midrule
			8$\times$8,  32 & 4$\times$4, 64 & 3$\times$3, 64 & 512 & k\\
			s. 4$\times$4 &   s. 2$\times$2 &   s. 1$\times$1 &  & \\
			p. same & p. same & p. same &&\\
			ReLU & ReLU & ReLU & ReLU & ReLU  \\
			\bottomrule
		\end{tabular}
		\captionof{table}{Network architecture}
		\label{tab:arc}
	\end{figure}
	
	
	\section{Tasks}
	\label{section:tasks}
	
	\subsection{Wrappers}
	The \texttt{FrameStack} wrappers is used to stack the last $k$ frames returning a lazy array, which is a structure much more memory efficient.

	The \texttt{ScaledFloatFrame} wrapper rescales the value of the pixels initially comprised between 0 and 255 between 0-1.
	
	The \texttt{MaxAndSkipEnv} wrapper is used to reduce the number of frame, hence the quantity of data to process. In order to do this, firstly it skips some frames of the game play. Then, for all the skipped frames, it continually repeats the same action, sums the rewards in order not to lose information and at the end, for each pixel of the frame, the maximum pixel value is chosen among all the frames skipped.
	 	 
	The \texttt{ClipRewardEnv} wrapper classifies the reward as $+1$, $0$ or $-1$ according its sign.

	
	\subsection{Online Q-network and Target Q-network}	
	As mention in Section \ref{section:agent}, the addition of the target network, the this used to generate targets in the Q-learning update, improves the stability of the presented method. In particular, every $C$ steps, the target network is updated with the online network parameters.
	This procedure adds a delay between the time between the time the network is updated and the time when the update affects the target, making divergence or oscillations much more unlikely.
	
	\subsection{$\epsilon$-greedy policy}
	%FIXME
	
	Acting according to an $\epsilon$-greedy policy ensure an adequate exploration of the environment in order to learn about potentially (could be better or worse) new sources of reward, instead of exploit the well-known sources of reward.
	
	For a given state $s$, the $\epsilon$-greedy policy with respect to Q chooses a random action with probability $\epsilon$, and an action $\arg \max_a Q(s, a)$ with probability $1 - \epsilon$.
	
	\subsection{Experiment 1}
	The first experiment includes a training phase the takes $2000000$ steps. 
	Root mean square prop (\texttt{RMSprop}) is used as optimiser.
	
	Figure \ref{fig:step-m1} shows a subsampling of the number of steps elapsed in each episode.
	
	\begin{figure}[htb]
		\centering
		\includegraphics[width=.8\linewidth]{../code/out/m1/img/step-per-episode.pdf}	
		\caption{Steps per episode of the first experiment}
		\label{fig:step-m1}
	\end{figure}

	It is clearly visible how the number of steps for episode increase over time. Since each episode corresponds to a "life of play", a greater number of steps per episode can be interpreted as an improvement in the network's ability to play the game without losing a life quickly. 
	\bigskip
	
	Figures \ref{fig:return-movingavg-m1} and \ref{fig:score-m1} show the return per episode, averaged over the last 30 episodes (moving average) to reduce noise and scores across 30 independent plays, that correspond to the sum of the return obtained across a sequence of 5 different episodes.
	
	\begin{figure}[htb]
		\begin{minipage}[b]{.49\textwidth}
			\centering
			\includegraphics[width=\linewidth]{../code/out/m1/img/score-moving-average.pdf}	
			\caption{Return per episode}
			\label{fig:return-movingavg-m1}
		\end{minipage}
		~
		\begin{minipage}[b]{.49\textwidth}
			\centering
			\includegraphics[width=\linewidth]{../code/out/m1/img/score.pdf}	
			\caption{Score}
			\label{fig:score-m1}
		\end{minipage}
	\end{figure}

	The game score obtained is $358.0$ that is, as expected, a little lower compared to the one achieved in literature, that is $401.2 (\pm 26.9)$, since the model is more complex.
	
	\bigskip
		
	At the end, in Figure \ref{fig:loss-m1} is shown a subsample of the temporal-difference error L($\theta$) averaged over the last 50 steps in order to reduce noise.
	\begin{figure}[htb]
		\centering
		\includegraphics[width=.8\linewidth]{../code/out/m1/img/loss-moving-average.pdf}	
		\caption{Loss of the first experiment}
		\label{fig:loss-m1}
	\end{figure} 


	% FIXME time
	It is also useful to estimate the remaining training time based on the average time that each step requires.
	
	\subsection{Experiment 2}
	The second experiment performed modifies only the value of the parameter $C$ with respect to the previous experiment. In particular, the target network is updated every $50000$ instead of $10000$.
	
	Figure \ref{fig:score-m1-m2} shows a comparison among the evaluation scores of the experiments.
	
	\begin{figure}[htb]
		\centering
		\includegraphics[width=.8\linewidth]{../code/out/m2/img/comparison/scores-comparison-m1-m2.pdf}	
		\caption{Score comparison}
		\label{fig:score-m1-m2}
	\end{figure}
	
	%FIXME
	It is clearly visible how … 
	\bigskip
	
	%FIXME
	time
	
	\subsection{Experiment 3}
	The third experiment repeats the training procedure with a different environment. The Atari game used in this case is \textit{StarGunner}.
	
	Figures \ref{fig:return-movingavg-m3} and \ref{fig:score-m3} show the return per episode, averaged over the last 30 episodes (moving average) to reduce noise and scores across 30 independent plays, that correspond to the sum of the return obtained across a sequence of 5 different episodes.
	
	\begin{figure}[htb]
		\begin{minipage}[b]{.49\textwidth}
			\centering
			\includegraphics[width=\linewidth]{../code/out/m3/img/score-moving-average.pdf}	
			\caption{Return per episode}
			\label{fig:return-movingavg-m3}
		\end{minipage}
		~
		\begin{minipage}[b]{.49\textwidth}
			\centering
			\includegraphics[width=\linewidth]{../code/out/m3/img/score.pdf}	
			\caption{Score}
			\label{fig:score-m3}
		\end{minipage}
	\end{figure}
	
	The game score obtained is $358.0$ that is, as expected, a little lower compared to the one achieved in literature, that is $401.2 (\pm 26.9)$, since the model is more complex.
	
	\bigskip
	
	At the end, in Figure \ref{fig:loss-m3} is shown a subsample of the temporal-difference error L($\theta$) averaged over the last 50 steps in order to reduce noise.
	\begin{figure}[htb]
		\centering
		\includegraphics[width=.8\linewidth]{../code/out/m3/img/loss-moving-average.pdf}	
		\caption{Loss of the third experiment}
		\label{fig:loss-m3}
	\end{figure} 
	
	
	% FIXME time
	It is also useful to estimate the remaining training time based on the average time that each step requires.
	
	\subsection{Experiment 4}
	The last experiment tries to improve the results obtained in the first experiment with the \texttt{Breakout} environment, using the data recorded from $10000$ gameplay steps to populate the replay buffer.
	In this case the model is trained for $300000$ steps.
	\bigskip

	Figures \ref{fig:return-movingavg-m4} and \ref{fig:score-m4} show the return per episode, averaged over the last 30 episodes (moving average) to reduce noise and scores across 30 independent plays, that correspond to the sum of the return obtained across a sequence of 5 different episodes.
	
	\begin{figure}[h]
		\begin{minipage}[b]{.49\textwidth}
			\centering
			\includegraphics[width=\linewidth]{../code/out/m4/img/score-moving-average.pdf}	
			\caption{Return per episode}
			\label{fig:return-movingavg-m4}
		\end{minipage}
		~
		\begin{minipage}[b]{.49\textwidth}
			\centering
			\includegraphics[width=\linewidth]{../code/out/m4/img/score.pdf}	
			\caption{Score}
			\label{fig:score-m4}
		\end{minipage}
	\end{figure}
	
	The game score obtained is $17.17$ that is similar to the one obtained in the first experiment at the same step, that is $19.47$.
	
	\bigskip
	
	In Figure \ref{fig:score-m1-m4} is shown a comparison among the evaluation scores of the first and the last experiments.
	
	\begin{figure}[htb]
		\centering
		\includegraphics[width=.8\linewidth]{../code/out/m4/img/comparison/scores-comparison-m1-m4.pdf}	
		\caption{Score comparison}
		\label{fig:score-m1-m4}
	\end{figure}
	
	
\end{document}
